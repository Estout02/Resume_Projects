Library Guide

This guide is provided as a supplement to the User Guide. It is recommended that all users start with the user guide then move onto this guide as the library file is created when cmake is run. If you have run the install instructions you will see that along with the msdscript executable there is also a library file name M\+S\+Dscriptlib.

Embed this library file in your application Using \#include {\ttfamily \char`\"{}\+M\+S\+Dscriptlib\char`\"{}}.

{\bfseries{Using the library}}

msdscript, takes a string and parses it into expressions, you can see the grammar that msdscript follows in the user guide. It follows all of the input rules that are covered in the user guide. The parser takes a string and parses it into expressions that can then be interpreted or optimized. The way that the command line tool works is as follows.

{\ttfamily P\+T\+R(\+Expr) e;}

{\ttfamily e = parse(std\+::cin);}

Parse takes an input stream that contains an expression, and returns the parsed representation of that expression. Throws {\ttfamily runtime\+\_\+error} for parse errors. This function is used to parse input.

After input has been parsed you will need to choose which mode to run; optimize, step, or interpret. This is done using the following commands.

For optimize mode\+:

{\ttfamily std\+::cout $<$$<$ e-\/$>$optimize()-\/$>$to\+String();}

For interpret mode\+:

{\ttfamily std\+::cout $<$$<$ e-\/$>$interp(\+Env\+::empty)-\/$>$to\+\_\+string();}

For step mode\+:

{\ttfamily std\+::cout $<$$<$ Step\+::interp\+\_\+by\+\_\+steps(e)-\/$>$to\+\_\+string();}

See the documentation for more in depth descriptions of the different expression types and more information about how they can be used. 